\documentclass[a4paper]{scrartcl}

\usepackage[T1]{fontenc}
\usepackage[utf8]{inputenc}
\usepackage[ukenglish,nswissgerman]{babel}
\usepackage[onehalfspacing]{setspace}
\usepackage{graphicx}

\setcounter{secnumdepth}{3}

\title{Sequenzen in \emph{Das Wohltemperierte Klavier, 1.~Band} von J. S. Bach}
\subtitle{Kontrapunktische Analysen von Modellen, Synkopendissonanzen und deren Verzierungen in ausgewählten Sequenzen aus den Fugen}
\subject{Bach Sequenzen}
\author{
	Wolfgang Drescher\thanks{Klasse: Dr. Felix Diergarten}\\
	Hochschule für Musik Freiburg
}
\date{\today}


\begin{document}

\maketitle

\tableofcontents

\section{Einleitung}

Lorem ipsum dolor sit amet, consectetur adipisicing elit, sed do eiusmod tempor incididunt ut labore et dolore magna aliqua. Ut enim ad minim veniam, quis nostrud exercitation ullamco laboris nisi ut aliquip ex ea commodo consequat. Duis aute irure dolor in reprehenderit in voluptate velit esse cillum dolore eu fugiat nulla pariatur. Excepteur sint occaecat cupidatat non proident, sunt in culpa qui officia deserunt mollit anim id est laborum.

\subsection{Problemstellung}

\subsection{Ziel dieses Artikels}

\subsection{Vorgehensweise}

\section{Fuge in g-moll, BWV 861, T.~24ff}

\begin{figure}[htbp]
	\centering
	\includegraphics{lilypond/g-moll/render/original}
	\caption{Fuge in g-moll, BWV 861, T.~24ff}
	\label{fig:bwv681-original}
\end{figure}

\end{document}
