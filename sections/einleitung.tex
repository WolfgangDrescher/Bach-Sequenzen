\section{Einleitung}

Im Musiktheorie Unterricht beschäftigen wir\footnote{Im gemeinsamen Musiktheorie-Unterricht von Adrian Nagel und mir, bei Prof. Dr. Felix Diergarten} uns in diesem Semester\footnote{Wintersemester 2017/2018} mit Fugen und versuchen neben dem eigenen nachkomponieren, angelehnt an Bachs Wohltemperiertes Clavier, auch mit wöchentlichen Analysen von Bachs Fugen aus dem ersten Band Satztechniken und Methoden zu finden, mit denen wir unsere eigenen Stilkopien an den Sound von Bach annähern können.
Im Zuge dessen sind uns immer wieder Stellen aufgefallen, die durch besondere Bezifferungen und Kontrapunktik einen sehr Bach typischen Klang erzeugen. Die eindrücklichsten dieser Stellen sollen in diesem Artikel aufgezeigt und analysiert werden.

\subsection{Problemstellung}

\subsection{Ziel dieses Artikels}

\subsection{Vorgehensweise}
