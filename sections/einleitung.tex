\section{Einleitung}

Im Musiktheorie-Unterricht beschäftigen wir\footnote{In der Gruppenstunde von Adrian Nagel und mir, bei Felix Diergarten} uns in diesem Semester\footnote{Wintersemester 2017/2018} mit Fugen und versuchen neben dem eigenen nach komponieren, angelehnt an Bachs Wohltemperiertes Clavier, auch mit Analysen von Bachs Fugen aus dem ersten Band Satztechniken und Methoden zu finden, mit denen wir unsere eigenen Stilkopien an den Klang Bachs annähern können.
Im Zuge dessen sind uns immer wieder Stellen aufgefallen, die durch besondere Bezifferungen und Kontrapunktik einen sehr Bach-typischen Klang erzeugen.
Die eindrücklichsten dieser Stellen sollen in diesem Artikel aufgezeigt und analysiert werden.


\subsection{Problemstellung}

Viele der von uns angesehenen Traktate geben ausführliche Informationen, wie eine Fugenexposition aufgebaut ist, besonders wie der der Comes eingerichtet werden muss, aber es liessen sich kaum Stellen finden, wie der weitere Verlauf einer Fuge aufgebaut ist.
Erwin Ratz schreibt in seiner Formenlehre\autocite[21]{ratz:formenlehre} von zwei Grundprinzipien in Instrumentalformen: \emph{fester Gefügtes} und \emph{locker Gefügtes}.
Er versucht zu zeigen, dass Beethoven nicht als Gegensatz zu Bach empfunden werden soll, sondern als organische Weiterentwicklung.
Während \emph{fester gefügten} Teile wie z.B. dem Hauptsatz einer Sonate oder der Exposition einer Fuge relativ klar definierten Kompositionsprinzipien folgen, bleiben, nicht nur bei Ratz, die Teile dazwischen unscharf erklärt.
Es ist deshalb besonders schwer, bei eigenen Stilkopien die Sequenzen und Zwischenspiele, seien diese komplett frei, nur harmonische Sequenzen\footnote{Damit meine ich Stellen, die nur harmonisch den Stationen einer Sequenz wie etwas einem Quintfall folgen, aber Rhythmisch nicht so aufgebaut sind, dass diese als regelmässige Sequenzierungen wahrgenommen werden, da keine Teile wörtlich auf einer anderen Stufe transponiert erklingen.}, oder tatsächlich wörtlich sequenzierte Abschnitte, näher am Vorbild der Bach Fugen nach zu komponieren, ohne dass sich diese Sequenzen eher anhören die Sequenzen aus den Triosonaten von Arcangelo Corelli.


\subsection{Ziel dieses Artikels}

An diesem Punkt soll dieser Artikel ansetzten.
Um ein besseres Verständnis dafür zu bekommen, wie genau Bach solche formal nicht streng definierten Stellen komponiert und warum diese den für Bach so typischen Tonfall haben, werden in diesem Artikel ausgewählte Sequenzen kontrapunktisch analysiert und versucht auf die üblichen Sequenzmodelle der Generalbass-Epoche zurückzuführen.
Ich hoffe dabei dem Leser Möglichkeiten mitzugeben, wie innerhalb einer Fuge die verschiedenen Durchführungen mit Sequenzen miteinander verbunden werden können und wie man diese durch Verzierungen, Variationen oder mit \emph{theatralischen Dissonanzauflösungen}\autocite[208-219]{heinichen:general-bass,menke:kontrapunkt2} in einen klanglichen Charakter bringen kann der näher an Bachs Stil ist als das blosse Sequenzmodell mit seinen üblichen Verzierungen.


\subsection{Vorgehensweise}

Um dem Leser für die eigene Komposition von Fugen im Stile Bachs Techniken mit auf den Weg zu geben und wie solche Sequenzen selbst nachgebaut werden können, habe ich bei den Analysen stets versucht das zu Grunde liegende Sequenzmodell auf dessen kontrapunktischen Kern zurückzuführen und dieses dann Stück für Stück wieder mit Verzierungen, \emph{Transitus} und Varianten anzureichern bis am Ende wieder die originale Komposition von Bach herauskommt.

Ich versuche damit keineswegs den Gedankengang von Bach beim Komponieren zu reproduzieren.
Vielmehr soll so für den Stilkopisten ein Weg gezeigt werden, wie durch Änderungen in der Generalbass-Bezifferung oder im Kontrapunkt so eine Stelle entsprechend simuliert werden kann.
